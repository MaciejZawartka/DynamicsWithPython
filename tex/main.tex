
% Default to the notebook output style

    


% Inherit from the specified cell style.




    
\documentclass[11pt]{article}

    
    
    \usepackage[T1]{fontenc}
    % Nicer default font (+ math font) than Computer Modern for most use cases
    \usepackage{mathpazo}

    % Basic figure setup, for now with no caption control since it's done
    % automatically by Pandoc (which extracts ![](path) syntax from Markdown).
    \usepackage{graphicx}
    % We will generate all images so they have a width \maxwidth. This means
    % that they will get their normal width if they fit onto the page, but
    % are scaled down if they would overflow the margins.
    \makeatletter
    \def\maxwidth{\ifdim\Gin@nat@width>\linewidth\linewidth
    \else\Gin@nat@width\fi}
    \makeatother
    \let\Oldincludegraphics\includegraphics
    % Set max figure width to be 80% of text width, for now hardcoded.
    \renewcommand{\includegraphics}[1]{\Oldincludegraphics[width=.8\maxwidth]{#1}}
    % Ensure that by default, figures have no caption (until we provide a
    % proper Figure object with a Caption API and a way to capture that
    % in the conversion process - todo).
    \usepackage{caption}
    \DeclareCaptionLabelFormat{nolabel}{}
    \captionsetup{labelformat=nolabel}

    \usepackage{adjustbox} % Used to constrain images to a maximum size 
    \usepackage{xcolor} % Allow colors to be defined
    \usepackage{enumerate} % Needed for markdown enumerations to work
    \usepackage{geometry} % Used to adjust the document margins
    \usepackage{amsmath} % Equations
    \usepackage{amssymb} % Equations
    \usepackage{textcomp} % defines textquotesingle
    % Hack from http://tex.stackexchange.com/a/47451/13684:
    \AtBeginDocument{%
        \def\PYZsq{\textquotesingle}% Upright quotes in Pygmentized code
    }
    \usepackage{upquote} % Upright quotes for verbatim code
    \usepackage{eurosym} % defines \euro
    \usepackage[mathletters]{ucs} % Extended unicode (utf-8) support
    \usepackage[utf8x]{inputenc} % Allow utf-8 characters in the tex document
    \usepackage{fancyvrb} % verbatim replacement that allows latex
    \usepackage{grffile} % extends the file name processing of package graphics 
                         % to support a larger range 
    % The hyperref package gives us a pdf with properly built
    % internal navigation ('pdf bookmarks' for the table of contents,
    % internal cross-reference links, web links for URLs, etc.)
    \usepackage{hyperref}
    \usepackage{longtable} % longtable support required by pandoc >1.10
    \usepackage{booktabs}  % table support for pandoc > 1.12.2
    \usepackage[inline]{enumitem} % IRkernel/repr support (it uses the enumerate* environment)
    \usepackage[normalem]{ulem} % ulem is needed to support strikethroughs (\sout)
                                % normalem makes italics be italics, not underlines
    \usepackage{mathrsfs}
    

    
    
    % Colors for the hyperref package
    \definecolor{urlcolor}{rgb}{0,.145,.698}
    \definecolor{linkcolor}{rgb}{.71,0.21,0.01}
    \definecolor{citecolor}{rgb}{.12,.54,.11}

    % ANSI colors
    \definecolor{ansi-black}{HTML}{3E424D}
    \definecolor{ansi-black-intense}{HTML}{282C36}
    \definecolor{ansi-red}{HTML}{E75C58}
    \definecolor{ansi-red-intense}{HTML}{B22B31}
    \definecolor{ansi-green}{HTML}{00A250}
    \definecolor{ansi-green-intense}{HTML}{007427}
    \definecolor{ansi-yellow}{HTML}{DDB62B}
    \definecolor{ansi-yellow-intense}{HTML}{B27D12}
    \definecolor{ansi-blue}{HTML}{208FFB}
    \definecolor{ansi-blue-intense}{HTML}{0065CA}
    \definecolor{ansi-magenta}{HTML}{D160C4}
    \definecolor{ansi-magenta-intense}{HTML}{A03196}
    \definecolor{ansi-cyan}{HTML}{60C6C8}
    \definecolor{ansi-cyan-intense}{HTML}{258F8F}
    \definecolor{ansi-white}{HTML}{C5C1B4}
    \definecolor{ansi-white-intense}{HTML}{A1A6B2}
    \definecolor{ansi-default-inverse-fg}{HTML}{FFFFFF}
    \definecolor{ansi-default-inverse-bg}{HTML}{000000}

    % commands and environments needed by pandoc snippets
    % extracted from the output of `pandoc -s`
    \providecommand{\tightlist}{%
      \setlength{\itemsep}{0pt}\setlength{\parskip}{0pt}}
    \DefineVerbatimEnvironment{Highlighting}{Verbatim}{commandchars=\\\{\}}
    % Add ',fontsize=\small' for more characters per line
    \newenvironment{Shaded}{}{}
    \newcommand{\KeywordTok}[1]{\textcolor[rgb]{0.00,0.44,0.13}{\textbf{{#1}}}}
    \newcommand{\DataTypeTok}[1]{\textcolor[rgb]{0.56,0.13,0.00}{{#1}}}
    \newcommand{\DecValTok}[1]{\textcolor[rgb]{0.25,0.63,0.44}{{#1}}}
    \newcommand{\BaseNTok}[1]{\textcolor[rgb]{0.25,0.63,0.44}{{#1}}}
    \newcommand{\FloatTok}[1]{\textcolor[rgb]{0.25,0.63,0.44}{{#1}}}
    \newcommand{\CharTok}[1]{\textcolor[rgb]{0.25,0.44,0.63}{{#1}}}
    \newcommand{\StringTok}[1]{\textcolor[rgb]{0.25,0.44,0.63}{{#1}}}
    \newcommand{\CommentTok}[1]{\textcolor[rgb]{0.38,0.63,0.69}{\textit{{#1}}}}
    \newcommand{\OtherTok}[1]{\textcolor[rgb]{0.00,0.44,0.13}{{#1}}}
    \newcommand{\AlertTok}[1]{\textcolor[rgb]{1.00,0.00,0.00}{\textbf{{#1}}}}
    \newcommand{\FunctionTok}[1]{\textcolor[rgb]{0.02,0.16,0.49}{{#1}}}
    \newcommand{\RegionMarkerTok}[1]{{#1}}
    \newcommand{\ErrorTok}[1]{\textcolor[rgb]{1.00,0.00,0.00}{\textbf{{#1}}}}
    \newcommand{\NormalTok}[1]{{#1}}
    
    % Additional commands for more recent versions of Pandoc
    \newcommand{\ConstantTok}[1]{\textcolor[rgb]{0.53,0.00,0.00}{{#1}}}
    \newcommand{\SpecialCharTok}[1]{\textcolor[rgb]{0.25,0.44,0.63}{{#1}}}
    \newcommand{\VerbatimStringTok}[1]{\textcolor[rgb]{0.25,0.44,0.63}{{#1}}}
    \newcommand{\SpecialStringTok}[1]{\textcolor[rgb]{0.73,0.40,0.53}{{#1}}}
    \newcommand{\ImportTok}[1]{{#1}}
    \newcommand{\DocumentationTok}[1]{\textcolor[rgb]{0.73,0.13,0.13}{\textit{{#1}}}}
    \newcommand{\AnnotationTok}[1]{\textcolor[rgb]{0.38,0.63,0.69}{\textbf{\textit{{#1}}}}}
    \newcommand{\CommentVarTok}[1]{\textcolor[rgb]{0.38,0.63,0.69}{\textbf{\textit{{#1}}}}}
    \newcommand{\VariableTok}[1]{\textcolor[rgb]{0.10,0.09,0.49}{{#1}}}
    \newcommand{\ControlFlowTok}[1]{\textcolor[rgb]{0.00,0.44,0.13}{\textbf{{#1}}}}
    \newcommand{\OperatorTok}[1]{\textcolor[rgb]{0.40,0.40,0.40}{{#1}}}
    \newcommand{\BuiltInTok}[1]{{#1}}
    \newcommand{\ExtensionTok}[1]{{#1}}
    \newcommand{\PreprocessorTok}[1]{\textcolor[rgb]{0.74,0.48,0.00}{{#1}}}
    \newcommand{\AttributeTok}[1]{\textcolor[rgb]{0.49,0.56,0.16}{{#1}}}
    \newcommand{\InformationTok}[1]{\textcolor[rgb]{0.38,0.63,0.69}{\textbf{\textit{{#1}}}}}
    \newcommand{\WarningTok}[1]{\textcolor[rgb]{0.38,0.63,0.69}{\textbf{\textit{{#1}}}}}
    
    
    % Define a nice break command that doesn't care if a line doesn't already
    % exist.
    \def\br{\hspace*{\fill} \\* }
    % Math Jax compatibility definitions
    \def\gt{>}
    \def\lt{<}
    \let\Oldtex\TeX
    \let\Oldlatex\LaTeX
    \renewcommand{\TeX}{\textrm{\Oldtex}}
    \renewcommand{\LaTeX}{\textrm{\Oldlatex}}
    % Document parameters
    % Document title
    \title{main}
    
    
    
    
    

    % Pygments definitions
    
\makeatletter
\def\PY@reset{\let\PY@it=\relax \let\PY@bf=\relax%
    \let\PY@ul=\relax \let\PY@tc=\relax%
    \let\PY@bc=\relax \let\PY@ff=\relax}
\def\PY@tok#1{\csname PY@tok@#1\endcsname}
\def\PY@toks#1+{\ifx\relax#1\empty\else%
    \PY@tok{#1}\expandafter\PY@toks\fi}
\def\PY@do#1{\PY@bc{\PY@tc{\PY@ul{%
    \PY@it{\PY@bf{\PY@ff{#1}}}}}}}
\def\PY#1#2{\PY@reset\PY@toks#1+\relax+\PY@do{#2}}

\expandafter\def\csname PY@tok@w\endcsname{\def\PY@tc##1{\textcolor[rgb]{0.73,0.73,0.73}{##1}}}
\expandafter\def\csname PY@tok@c\endcsname{\let\PY@it=\textit\def\PY@tc##1{\textcolor[rgb]{0.25,0.50,0.50}{##1}}}
\expandafter\def\csname PY@tok@cp\endcsname{\def\PY@tc##1{\textcolor[rgb]{0.74,0.48,0.00}{##1}}}
\expandafter\def\csname PY@tok@k\endcsname{\let\PY@bf=\textbf\def\PY@tc##1{\textcolor[rgb]{0.00,0.50,0.00}{##1}}}
\expandafter\def\csname PY@tok@kp\endcsname{\def\PY@tc##1{\textcolor[rgb]{0.00,0.50,0.00}{##1}}}
\expandafter\def\csname PY@tok@kt\endcsname{\def\PY@tc##1{\textcolor[rgb]{0.69,0.00,0.25}{##1}}}
\expandafter\def\csname PY@tok@o\endcsname{\def\PY@tc##1{\textcolor[rgb]{0.40,0.40,0.40}{##1}}}
\expandafter\def\csname PY@tok@ow\endcsname{\let\PY@bf=\textbf\def\PY@tc##1{\textcolor[rgb]{0.67,0.13,1.00}{##1}}}
\expandafter\def\csname PY@tok@nb\endcsname{\def\PY@tc##1{\textcolor[rgb]{0.00,0.50,0.00}{##1}}}
\expandafter\def\csname PY@tok@nf\endcsname{\def\PY@tc##1{\textcolor[rgb]{0.00,0.00,1.00}{##1}}}
\expandafter\def\csname PY@tok@nc\endcsname{\let\PY@bf=\textbf\def\PY@tc##1{\textcolor[rgb]{0.00,0.00,1.00}{##1}}}
\expandafter\def\csname PY@tok@nn\endcsname{\let\PY@bf=\textbf\def\PY@tc##1{\textcolor[rgb]{0.00,0.00,1.00}{##1}}}
\expandafter\def\csname PY@tok@ne\endcsname{\let\PY@bf=\textbf\def\PY@tc##1{\textcolor[rgb]{0.82,0.25,0.23}{##1}}}
\expandafter\def\csname PY@tok@nv\endcsname{\def\PY@tc##1{\textcolor[rgb]{0.10,0.09,0.49}{##1}}}
\expandafter\def\csname PY@tok@no\endcsname{\def\PY@tc##1{\textcolor[rgb]{0.53,0.00,0.00}{##1}}}
\expandafter\def\csname PY@tok@nl\endcsname{\def\PY@tc##1{\textcolor[rgb]{0.63,0.63,0.00}{##1}}}
\expandafter\def\csname PY@tok@ni\endcsname{\let\PY@bf=\textbf\def\PY@tc##1{\textcolor[rgb]{0.60,0.60,0.60}{##1}}}
\expandafter\def\csname PY@tok@na\endcsname{\def\PY@tc##1{\textcolor[rgb]{0.49,0.56,0.16}{##1}}}
\expandafter\def\csname PY@tok@nt\endcsname{\let\PY@bf=\textbf\def\PY@tc##1{\textcolor[rgb]{0.00,0.50,0.00}{##1}}}
\expandafter\def\csname PY@tok@nd\endcsname{\def\PY@tc##1{\textcolor[rgb]{0.67,0.13,1.00}{##1}}}
\expandafter\def\csname PY@tok@s\endcsname{\def\PY@tc##1{\textcolor[rgb]{0.73,0.13,0.13}{##1}}}
\expandafter\def\csname PY@tok@sd\endcsname{\let\PY@it=\textit\def\PY@tc##1{\textcolor[rgb]{0.73,0.13,0.13}{##1}}}
\expandafter\def\csname PY@tok@si\endcsname{\let\PY@bf=\textbf\def\PY@tc##1{\textcolor[rgb]{0.73,0.40,0.53}{##1}}}
\expandafter\def\csname PY@tok@se\endcsname{\let\PY@bf=\textbf\def\PY@tc##1{\textcolor[rgb]{0.73,0.40,0.13}{##1}}}
\expandafter\def\csname PY@tok@sr\endcsname{\def\PY@tc##1{\textcolor[rgb]{0.73,0.40,0.53}{##1}}}
\expandafter\def\csname PY@tok@ss\endcsname{\def\PY@tc##1{\textcolor[rgb]{0.10,0.09,0.49}{##1}}}
\expandafter\def\csname PY@tok@sx\endcsname{\def\PY@tc##1{\textcolor[rgb]{0.00,0.50,0.00}{##1}}}
\expandafter\def\csname PY@tok@m\endcsname{\def\PY@tc##1{\textcolor[rgb]{0.40,0.40,0.40}{##1}}}
\expandafter\def\csname PY@tok@gh\endcsname{\let\PY@bf=\textbf\def\PY@tc##1{\textcolor[rgb]{0.00,0.00,0.50}{##1}}}
\expandafter\def\csname PY@tok@gu\endcsname{\let\PY@bf=\textbf\def\PY@tc##1{\textcolor[rgb]{0.50,0.00,0.50}{##1}}}
\expandafter\def\csname PY@tok@gd\endcsname{\def\PY@tc##1{\textcolor[rgb]{0.63,0.00,0.00}{##1}}}
\expandafter\def\csname PY@tok@gi\endcsname{\def\PY@tc##1{\textcolor[rgb]{0.00,0.63,0.00}{##1}}}
\expandafter\def\csname PY@tok@gr\endcsname{\def\PY@tc##1{\textcolor[rgb]{1.00,0.00,0.00}{##1}}}
\expandafter\def\csname PY@tok@ge\endcsname{\let\PY@it=\textit}
\expandafter\def\csname PY@tok@gs\endcsname{\let\PY@bf=\textbf}
\expandafter\def\csname PY@tok@gp\endcsname{\let\PY@bf=\textbf\def\PY@tc##1{\textcolor[rgb]{0.00,0.00,0.50}{##1}}}
\expandafter\def\csname PY@tok@go\endcsname{\def\PY@tc##1{\textcolor[rgb]{0.53,0.53,0.53}{##1}}}
\expandafter\def\csname PY@tok@gt\endcsname{\def\PY@tc##1{\textcolor[rgb]{0.00,0.27,0.87}{##1}}}
\expandafter\def\csname PY@tok@err\endcsname{\def\PY@bc##1{\setlength{\fboxsep}{0pt}\fcolorbox[rgb]{1.00,0.00,0.00}{1,1,1}{\strut ##1}}}
\expandafter\def\csname PY@tok@kc\endcsname{\let\PY@bf=\textbf\def\PY@tc##1{\textcolor[rgb]{0.00,0.50,0.00}{##1}}}
\expandafter\def\csname PY@tok@kd\endcsname{\let\PY@bf=\textbf\def\PY@tc##1{\textcolor[rgb]{0.00,0.50,0.00}{##1}}}
\expandafter\def\csname PY@tok@kn\endcsname{\let\PY@bf=\textbf\def\PY@tc##1{\textcolor[rgb]{0.00,0.50,0.00}{##1}}}
\expandafter\def\csname PY@tok@kr\endcsname{\let\PY@bf=\textbf\def\PY@tc##1{\textcolor[rgb]{0.00,0.50,0.00}{##1}}}
\expandafter\def\csname PY@tok@bp\endcsname{\def\PY@tc##1{\textcolor[rgb]{0.00,0.50,0.00}{##1}}}
\expandafter\def\csname PY@tok@fm\endcsname{\def\PY@tc##1{\textcolor[rgb]{0.00,0.00,1.00}{##1}}}
\expandafter\def\csname PY@tok@vc\endcsname{\def\PY@tc##1{\textcolor[rgb]{0.10,0.09,0.49}{##1}}}
\expandafter\def\csname PY@tok@vg\endcsname{\def\PY@tc##1{\textcolor[rgb]{0.10,0.09,0.49}{##1}}}
\expandafter\def\csname PY@tok@vi\endcsname{\def\PY@tc##1{\textcolor[rgb]{0.10,0.09,0.49}{##1}}}
\expandafter\def\csname PY@tok@vm\endcsname{\def\PY@tc##1{\textcolor[rgb]{0.10,0.09,0.49}{##1}}}
\expandafter\def\csname PY@tok@sa\endcsname{\def\PY@tc##1{\textcolor[rgb]{0.73,0.13,0.13}{##1}}}
\expandafter\def\csname PY@tok@sb\endcsname{\def\PY@tc##1{\textcolor[rgb]{0.73,0.13,0.13}{##1}}}
\expandafter\def\csname PY@tok@sc\endcsname{\def\PY@tc##1{\textcolor[rgb]{0.73,0.13,0.13}{##1}}}
\expandafter\def\csname PY@tok@dl\endcsname{\def\PY@tc##1{\textcolor[rgb]{0.73,0.13,0.13}{##1}}}
\expandafter\def\csname PY@tok@s2\endcsname{\def\PY@tc##1{\textcolor[rgb]{0.73,0.13,0.13}{##1}}}
\expandafter\def\csname PY@tok@sh\endcsname{\def\PY@tc##1{\textcolor[rgb]{0.73,0.13,0.13}{##1}}}
\expandafter\def\csname PY@tok@s1\endcsname{\def\PY@tc##1{\textcolor[rgb]{0.73,0.13,0.13}{##1}}}
\expandafter\def\csname PY@tok@mb\endcsname{\def\PY@tc##1{\textcolor[rgb]{0.40,0.40,0.40}{##1}}}
\expandafter\def\csname PY@tok@mf\endcsname{\def\PY@tc##1{\textcolor[rgb]{0.40,0.40,0.40}{##1}}}
\expandafter\def\csname PY@tok@mh\endcsname{\def\PY@tc##1{\textcolor[rgb]{0.40,0.40,0.40}{##1}}}
\expandafter\def\csname PY@tok@mi\endcsname{\def\PY@tc##1{\textcolor[rgb]{0.40,0.40,0.40}{##1}}}
\expandafter\def\csname PY@tok@il\endcsname{\def\PY@tc##1{\textcolor[rgb]{0.40,0.40,0.40}{##1}}}
\expandafter\def\csname PY@tok@mo\endcsname{\def\PY@tc##1{\textcolor[rgb]{0.40,0.40,0.40}{##1}}}
\expandafter\def\csname PY@tok@ch\endcsname{\let\PY@it=\textit\def\PY@tc##1{\textcolor[rgb]{0.25,0.50,0.50}{##1}}}
\expandafter\def\csname PY@tok@cm\endcsname{\let\PY@it=\textit\def\PY@tc##1{\textcolor[rgb]{0.25,0.50,0.50}{##1}}}
\expandafter\def\csname PY@tok@cpf\endcsname{\let\PY@it=\textit\def\PY@tc##1{\textcolor[rgb]{0.25,0.50,0.50}{##1}}}
\expandafter\def\csname PY@tok@c1\endcsname{\let\PY@it=\textit\def\PY@tc##1{\textcolor[rgb]{0.25,0.50,0.50}{##1}}}
\expandafter\def\csname PY@tok@cs\endcsname{\let\PY@it=\textit\def\PY@tc##1{\textcolor[rgb]{0.25,0.50,0.50}{##1}}}

\def\PYZbs{\char`\\}
\def\PYZus{\char`\_}
\def\PYZob{\char`\{}
\def\PYZcb{\char`\}}
\def\PYZca{\char`\^}
\def\PYZam{\char`\&}
\def\PYZlt{\char`\<}
\def\PYZgt{\char`\>}
\def\PYZsh{\char`\#}
\def\PYZpc{\char`\%}
\def\PYZdl{\char`\$}
\def\PYZhy{\char`\-}
\def\PYZsq{\char`\'}
\def\PYZdq{\char`\"}
\def\PYZti{\char`\~}
% for compatibility with earlier versions
\def\PYZat{@}
\def\PYZlb{[}
\def\PYZrb{]}
\makeatother


    % Exact colors from NB
    \definecolor{incolor}{rgb}{0.0, 0.0, 0.5}
    \definecolor{outcolor}{rgb}{0.545, 0.0, 0.0}



    
    % Prevent overflowing lines due to hard-to-break entities
    \sloppy 
    % Setup hyperref package
    \hypersetup{
      breaklinks=true,  % so long urls are correctly broken across lines
      colorlinks=true,
      urlcolor=urlcolor,
      linkcolor=linkcolor,
      citecolor=citecolor,
      }
    % Slightly bigger margins than the latex defaults
    
    \geometry{verbose,tmargin=1in,bmargin=1in,lmargin=1in,rmargin=1in}
    
    

    \begin{document}
    
    
    \maketitle
    
    

    
    Pisanie programu w języku Python wersji 3.6 oraz w nowszych,
rozpoczynamy od zaimportowania potrzebnych bibliotek.

    \begin{Verbatim}[commandchars=\\\{\}]
{\color{incolor}In [{\color{incolor}74}]:} \PY{c+c1}{\PYZsh{} DEKLARACJA BIBLIOTEK}
         \PY{k+kn}{import} \PY{n+nn}{numpy} \PY{k}{as} \PY{n+nn}{np}    \PY{c+c1}{\PYZsh{}obliczenia numeryczne, operacje na macierzach}
\end{Verbatim}

    Biblioteka NumPy (z ang. Numeric Python) to podstawowy zestaw narzędzi
inyżnierskich języka Python wspomagającym zaawasowane obliczenia
numeryczne, w szczególności operacje na macierzach. W powyższej linijce
kodu zaimportwana została biblioteka "numpy" do naszego programu.
Używanie nazwy "numpy" w kodzie, przy częstym jej wystepowaniu może być
uciążliwe. Stąd zastosowany został alias "np". Dzięki niemu, możemy
precyzować w kodzie, że dana funkcja lub metoda ma być zaimporotwana z
biblioteki numpy. Sam sposób odwoływania sie przy pomocy aliasu będzie
wyjaśniony później. Powyższą linijkę można dosłownie rozumieć jako
"Python, zaimportuj bibliotekę NumPy oraz nadaj jej pseudonim np". Warto
tutaj wspomnieć o własności języka Python jaką jest rozróżnianie
wielkości liter (z ang. Case sensitivity). Próba zaimporotowania
biblioteki NumPy z użyciem dużych i małych liter spowoduje powstanie
błędu przedstawionego poniżej.
https://docs.scipy.org/doc/numpy/reference/

    \begin{Verbatim}[commandchars=\\\{\}]
{\color{incolor}In [{\color{incolor}75}]:} \PY{k+kn}{import} \PY{n+nn}{NumPy} \PY{k}{as} \PY{n+nn}{np}
\end{Verbatim}

    \begin{Verbatim}[commandchars=\\\{\}]

        ---------------------------------------------------------------------------

        ModuleNotFoundError                       Traceback (most recent call last)

        <ipython-input-75-ec0b775bee05> in <module>()
    ----> 1 import NumPy as np
    

        ModuleNotFoundError: No module named 'NumPy'

    \end{Verbatim}

    \begin{Verbatim}[commandchars=\\\{\}]
{\color{incolor}In [{\color{incolor}76}]:} \PY{k+kn}{from} \PY{n+nn}{math} \PY{k}{import} \PY{n}{sin}\PY{p}{,} \PY{n}{cos}\PY{p}{,} \PY{n}{pi}    \PY{c+c1}{\PYZsh{}funkcje uzyte do wymuszen, wartosc liczby Pi}
\end{Verbatim}

    W przypadku, w którym wiemy dokładnie co chcemy zaimportować z danej
biblioteki, możemy odwołać się do konkretnej funkcjonalności. Tak jest
właśnie w przypadku kolejnej linii kodu. Z biblioteki \emph{math}
importujemy funkcje odpowiedzialne za wygenerowanie przebiegów sinusa i
cosinusa. W tym przypadku nie nadajemy żadnego aliasu.
https://docs.python.org/3/library/math.html.

    Dobrym nawykiem początkującego programisty jest używanie komentarzy. W
języku Python komentarz jest dodawany poprzez użycie symbolu "\#".
Kompilator pomija znaki znajdujące się po tym symbolu. W naszym
przypadku został dodany komentarz informujacy do czego użyta zostanie
dana biblioteka. Pomaga to na etapie "czyszczenia kodu", gdzie po
wypróbowaniu kilku bibliotek, zdecydujemy się w końcu na którąś i
będziemy chcieli inne (niewykorzystane) usunąć.

    Do naszego programu potrzebne jeszcze będą poniższe biblioteki. W sposób
skrócony ich funkcjonalność opisują komentarze. Na etapie wyboru
bibliotek warto jest zapoznać się z ich funkcjonalnościami. Szczegółowe
informacje można znaleźć wpisując w wyszukiwarkę nazwę biblioteki oraz
słowo "reference". Przykładowe strony użytych bibliotek:
https://docs.python.org/3/library/math.html
https://docs.scipy.org/doc/numpy/reference/ https://matplotlib.org

    \begin{Verbatim}[commandchars=\\\{\}]
{\color{incolor}In [{\color{incolor} }]:} 
\end{Verbatim}

    \begin{Verbatim}[commandchars=\\\{\}]
{\color{incolor}In [{\color{incolor}77}]:} \PY{k+kn}{from} \PY{n+nn}{scipy}\PY{n+nn}{.}\PY{n+nn}{linalg} \PY{k}{import} \PY{n}{eigh}         \PY{c+c1}{\PYZsh{}analiza harmoniczna }
         \PY{k+kn}{from} \PY{n+nn}{numpy}\PY{n+nn}{.}\PY{n+nn}{linalg} \PY{k}{import} \PY{n}{inv}          \PY{c+c1}{\PYZsh{}odwracanie macierzy   }
         \PY{k+kn}{from} \PY{n+nn}{matplotlib} \PY{k}{import} \PY{n}{pyplot} \PY{k}{as} \PY{n}{plot} \PY{c+c1}{\PYZsh{}rysowanie wykresow}
\end{Verbatim}

    Zdefiniujemy teraz konstrukcję naszego programu używając komentarzy:

    \begin{Verbatim}[commandchars=\\\{\}]
{\color{incolor}In [{\color{incolor}78}]:} \PY{c+c1}{\PYZsh{} Funkcje wlasne}
         
         \PY{c+c1}{\PYZsh{} Zdefiniowanie parametrow}
         
         \PY{c+c1}{\PYZsh{} Budowanie macierzy}
         
         \PY{c+c1}{\PYZsh{} Numeryczne calkowanie rownan ruchu}
         
         \PY{c+c1}{\PYZsh{} Wizualizacja wynikow}
\end{Verbatim}

    Właściwe programowanie rozpoczniemy od definiowania zmiennych. Funkcje
zostaną dodane później. Definiowanie funkcji na początku programu pomaga
odnaleźć je w przypadku chęci zrozumienia ich działania.

    \begin{Verbatim}[commandchars=\\\{\}]
{\color{incolor}In [{\color{incolor}79}]:} \PY{n}{Mzred} \PY{o}{=} \PY{l+m+mf}{1.0}
         \PY{n}{Jzred} \PY{o}{=} \PY{l+m+mf}{1.0}
         \PY{n}{k} \PY{o}{=} \PY{l+m+mf}{1.0}
         \PY{n}{b} \PY{o}{=} \PY{l+m+mf}{0.50}
         \PY{n}{L1} \PY{o}{=} \PY{l+m+mf}{1.0}
         \PY{n}{L2} \PY{o}{=} \PY{l+m+mf}{2.0} 
         \PY{n}{y1} \PY{o}{=} \PY{l+m+mf}{0.0}
         \PY{n}{y2} \PY{o}{=} \PY{l+m+mf}{0.1}
         \PY{n}{y3} \PY{o}{=} \PY{l+m+mf}{0.3}
         \PY{n}{y4} \PY{o}{=} \PY{l+m+mf}{0.4}
         \PY{n}{dof} \PY{o}{=} \PY{l+m+mi}{2}
         \PY{n}{vkm} \PY{o}{=} \PY{l+m+mf}{10.0} \PY{c+c1}{\PYZsh{}predkosc pojazdu}
         \PY{n}{t0} \PY{o}{=} \PY{l+m+mf}{0.0}   \PY{c+c1}{\PYZsh{}poczatek symulacji}
         \PY{n}{tk} \PY{o}{=} \PY{l+m+mf}{60.0} \PY{c+c1}{\PYZsh{}koniec symulacji}
         \PY{n}{ds} \PY{o}{=} \PY{l+m+mf}{0.01} \PY{c+c1}{\PYZsh{}kork podziału drogi}
\end{Verbatim}

    Są to wielkości pierwotne- wprowadzane przez użytkownika, opisujące
badany obiekt oraz jego otocznenie mające na ten obiekt wpływ. W
analizowanym przypadku, ów wpływ opisywany będzie jako nierówność
terenu. Poczynione zostało założenie, że krążki nośne zawieszenia
pojazdu nie odrywają się od podłoża, a przebieg terenu drogi odpowiada
przemieszczeniom środków krążków. W tym celu musimy wykonać dodatkowe
obliczenia. Rozpoczniemy od przeliczenia prędkości pojazdu z
\emph{{[}km/h{]}} na \emph{{[}m/s{]}}:

    \begin{Verbatim}[commandchars=\\\{\}]
{\color{incolor}In [{\color{incolor}80}]:} \PY{n}{v} \PY{o}{=} \PY{n}{vkm} \PY{o}{*} \PY{l+m+mi}{1000} \PY{o}{/} \PY{l+m+mi}{3600}
         \PY{n+nb}{print}\PY{p}{(}\PY{n}{v}\PY{p}{)}
\end{Verbatim}

    \begin{Verbatim}[commandchars=\\\{\}]
2.7777777777777777

    \end{Verbatim}

    Skoro środki krążków, będą się przemieszczały według profilu drogi,
musimy tak dobrać krok czasowy symulacji, aby w każdej chwili czasowej
symulacji \emph{t} należącej do wektora T∈(0, Tk) można było odczytać
pozycję środka krążka. W tym celu, krok czasowy wyliczymy biorąc pod
uwagę prędkość przemieszaczania się pojazdu oraz krok podziału drogi,
przyjęte podczas deklarowania zmiennych. Wykorzystany zostanie wzór v =
s/t

    \begin{Verbatim}[commandchars=\\\{\}]
{\color{incolor}In [{\color{incolor}81}]:} \PY{n}{dt} \PY{o}{=} \PY{n}{ds} \PY{o}{/} \PY{n}{v}
         \PY{n+nb}{print}\PY{p}{(}\PY{n}{dt}\PY{p}{)}
\end{Verbatim}

    \begin{Verbatim}[commandchars=\\\{\}]
0.0036000000000000003

    \end{Verbatim}

    Z punktu widzenia symulacji krok czasowy stanowi ważną zmienną mającą
wpływ na poprawność wyniku oraz czas wykonywania się obliczeń.
\textbf{tutaj dodać rysunek wykresu czas symulacji, liczba elementów i
dokładność}

    Uważni czytelnicy na pewno zauważyli, że na krok czasowy \emph{dt} wpływ
mają prędkość pojazdu oraz krok dyskretyzacji drogi, czyli zmienne
definiowane przez programistę. Jeśli \emph{v} jest prędkością określoną
przez konkrente zagadnienie, to jedynym parametrem sterującym zostaje
krok dyskretyzacji drogi.

    Teraz już na podstawie otrzymanych danych możemy zdefiniować wektor
czasu trwania symulacji \textbf{T} . Wykorzystana do tego celu zostanie
instrukcja arange pochodząca z biblioteki NumPy. Przyjżyjmy się sposobie
jej deklaracji w dokumentacji biblioteki: tutaj rysunek screen czy coś z
linka. Tworzy on wektor rozpoczynając od wartości \emph{t0} stosując
krok \emph{dt} do wartości końcowej \emph{tk}, przy czym nie zawiera
ostatniej wartości. Stąd do warunku końca dodamy jednokrotność
\emph{dt}.

    \begin{Verbatim}[commandchars=\\\{\}]
{\color{incolor}In [{\color{incolor}82}]:} \PY{n}{T} \PY{o}{=} \PY{n}{np}\PY{o}{.}\PY{n}{arange}\PY{p}{(}\PY{n}{t0}\PY{p}{,} \PY{n}{tk}\PY{o}{+}\PY{n}{dt}\PY{p}{,} \PY{n}{dt}\PY{p}{,} \PY{n}{dtype} \PY{o}{=} \PY{n+nb}{float} \PY{p}{)}
         \PY{n+nb}{print}\PY{p}{(}\PY{n}{T}\PY{p}{)}
\end{Verbatim}

    \begin{Verbatim}[commandchars=\\\{\}]
[0.00000e+00 3.60000e-03 7.20000e-03 {\ldots} 5.99940e+01 5.99976e+01
 6.00012e+01]

    \end{Verbatim}

    Ostatni wyraz macierzy przekracza nieznacznie czas, który założyliśmy.
Wynika to z błędu reprezentacji liczb przez komputer. Wartość \emph{tk}
jest przekroczona znacznie mniej niż wynosi krok \emph{dt} więc,
zastosujemy przypisanie wartości do ostatniej komórki macierzy, stosująć
indeksowanie z minusem.

    \begin{Verbatim}[commandchars=\\\{\}]
{\color{incolor}In [{\color{incolor}83}]:} \PY{n}{T}\PY{p}{[}\PY{o}{\PYZhy{}}\PY{l+m+mi}{1}\PY{p}{]} \PY{o}{=} \PY{n}{tk}
         \PY{n+nb}{print}\PY{p}{(}\PY{n}{T}\PY{p}{)}
         \PY{n+nb}{print}\PY{p}{(}\PY{n}{np}\PY{o}{.}\PY{n}{size}\PY{p}{(}\PY{n}{T}\PY{p}{)}\PY{p}{)}
\end{Verbatim}

    \begin{Verbatim}[commandchars=\\\{\}]
[0.00000e+00 3.60000e-03 7.20000e-03 {\ldots} 5.99940e+01 5.99976e+01
 6.00000e+01]
16668

    \end{Verbatim}

    Środki krążków są odpowiednio oddalone od środka ciężkości o długości
\emph{L1} oraz \emph{L2}. Co za tym idzie mają kontakt z podłożem w
różnych miejscach w jednej chwili czasowej. W związku z tym należy
wyznaczyć przesunięcie między kolejnymi próbkami, które należy odczytać
dla danego krążka: \textbf{dodać rysunek z ideą}

    \begin{Verbatim}[commandchars=\\\{\}]
{\color{incolor}In [{\color{incolor}84}]:} \PY{n}{phase} \PY{o}{=} \PY{n+nb}{round} \PY{p}{(}\PY{p}{(}\PY{n}{L2}\PY{o}{\PYZhy{}}\PY{n}{L1}\PY{p}{)}\PY{o}{/}\PY{n}{ds}\PY{p}{)}
         \PY{n+nb}{print}\PY{p}{(}\PY{n}{phase}\PY{p}{)}
\end{Verbatim}

    \begin{Verbatim}[commandchars=\\\{\}]
100

    \end{Verbatim}

    Język Python sam rozpoznaje typ danych, jakiego chcemy użyć. W
Internecie dostępnych jest wiele opracowań typów danych języka Python,
stąd nie będą one omawiane w tym artykule. Warto jednak wspomnieć o
sposobie na wymuszenie typu danych float w miejscu gdzie interpreter
oraz kompilator zdecydowały by się na liczbę całkowitą. Wystarczy dodać
po wartości ".0". Tak w języku Python jako separator dziesiętny używamy
kropki '.'. Typ danych można sprawdzić wywołując polecenie:

    \begin{Verbatim}[commandchars=\\\{\}]
{\color{incolor}In [{\color{incolor}85}]:} \PY{n+nb}{type}\PY{p}{(}\PY{n}{Mzred}\PY{p}{)}
\end{Verbatim}

\begin{Verbatim}[commandchars=\\\{\}]
{\color{outcolor}Out[{\color{outcolor}85}]:} float
\end{Verbatim}
            
    Na podstawie obliczonych przez równań (dodać odnośnik do równań)
zaimplementujemy teraz macierze w naszym programie. Przypomnijmy
oznaczenia M - macierz mas i bezwładności, K - macierz sprężystości, B -
macierz tłumienia, F - macierz wyrażeń wolnych I- macierz jednostkowa.
Do implementacji macierzy posłużymy się biblioteką NumPy. Przypomnijmy,
że nadalismy jej alias "np". Stąd macierz mas i bezwładności przyjmie
postać:

    \begin{Verbatim}[commandchars=\\\{\}]
{\color{incolor}In [{\color{incolor}86}]:} \PY{n}{M} \PY{o}{=} \PY{n}{np}\PY{o}{.}\PY{n}{array}\PY{p}{(}\PY{p}{[}\PY{p}{[}\PY{n}{Mzred}    \PY{p}{,}   \PY{l+m+mi}{0}\PY{p}{]}\PY{p}{,}\PY{p}{[}\PY{l+m+mi}{0}    \PY{p}{,} \PY{n}{Jzred}\PY{p}{]}\PY{p}{]}\PY{p}{)}   \PY{c+c1}{\PYZsh{}budowanie macierzy mas i bezwladnosci}
         \PY{n+nb}{print}\PY{p}{(}\PY{n}{M}\PY{p}{)}
\end{Verbatim}

    \begin{Verbatim}[commandchars=\\\{\}]
[[1. 0.]
 [0. 1.]]

    \end{Verbatim}

    Przyjrzymy się temu poleceniu. Początek instrukcji "np.array()" odwołuję
się do funkcjonalności biblioteki NumPy i każde odnaleźć polecenie
"array", które tworzy macierz na podstawie dostarczonych danych. Więcej
o tej metodzie możesz przeczytać tutaj:
https://docs.scipy.org/doc/numpy/reference/generated/numpy.array.html.
Następnie, wewnątrz okrągłego nawiasu mamy dwa nawiasy kwadratowe.
Patrząc szerzej zauważysz, że są one użyte w następującym ułożeniu: "{[}
{[}{]} , {[}{]} {]}". Wewnętrzne nawiasy określają zakres jednego
wiersza i przedzielone są przecinkiem. Zewnętrzny nawias określa cała
macierz. Wewnętrzne nawiasy zawierają poszeczególne elementy macierzy.
Jeśli na tym etapie pojawił Ci się błąd, sprawdź czy zgadza Ci się
liczba nawiasów. Większość edytorów tekstu ze sprawdzaniem składni
języka Python podpowiada, który nawias nie ma pary.

    W identyczny sposób zbudowane zostają pozostałe macierze:

    \begin{Verbatim}[commandchars=\\\{\}]
{\color{incolor}In [{\color{incolor}87}]:} \PY{n}{K} \PY{o}{=} \PY{n}{np}\PY{o}{.}\PY{n}{array}\PY{p}{(} \PY{p}{[}    \PY{p}{[}\PY{l+m+mi}{4} \PY{o}{*} \PY{n}{k}\PY{p}{,} \PY{l+m+mi}{0}\PY{p}{]}\PY{p}{,}\PY{p}{[}\PY{l+m+mi}{0}\PY{p}{,} \PY{l+m+mi}{2}\PY{o}{*}\PY{n}{k} \PY{o}{*} \PY{p}{(}\PY{n}{L1}\PY{o}{*}\PY{o}{*}\PY{l+m+mi}{2}\PY{p}{)} \PY{o}{+} \PY{l+m+mi}{2}\PY{o}{*}\PY{n}{k} \PY{o}{*} \PY{p}{(}\PY{n}{L2}\PY{o}{*}\PY{o}{*}\PY{l+m+mi}{2}\PY{p}{)}  \PY{p}{]}   \PY{p}{]} \PY{p}{)}
         \PY{n}{B} \PY{o}{=} \PY{n}{np}\PY{o}{.}\PY{n}{array}\PY{p}{(} \PY{p}{[}    \PY{p}{[}\PY{l+m+mi}{4} \PY{o}{*} \PY{n}{b}\PY{p}{,} \PY{l+m+mi}{0}\PY{p}{]}\PY{p}{,}\PY{p}{[}\PY{l+m+mi}{0}\PY{p}{,} \PY{l+m+mi}{2}\PY{o}{*}\PY{n}{b} \PY{o}{*} \PY{p}{(}\PY{n}{L1}\PY{o}{*}\PY{o}{*}\PY{l+m+mi}{2}\PY{p}{)} \PY{o}{+} \PY{l+m+mi}{2}\PY{o}{*}\PY{n}{b} \PY{o}{*} \PY{p}{(}\PY{n}{L2}\PY{o}{*}\PY{o}{*}\PY{l+m+mi}{2}\PY{p}{)}  \PY{p}{]}   \PY{p}{]} \PY{p}{)}
         \PY{n}{D} \PY{o}{=} \PY{n}{np}\PY{o}{.}\PY{n}{array}\PY{p}{(} \PY{p}{[}\PY{p}{[}\PY{n}{k}\PY{o}{*}\PY{p}{(}\PY{n}{y1} \PY{o}{+} \PY{n}{y2} \PY{o}{+} \PY{n}{y3} \PY{o}{+} \PY{n}{y4}\PY{p}{)}\PY{p}{]}\PY{p}{,}\PY{p}{[}\PY{n}{k} \PY{o}{*}\PY{p}{(}\PY{o}{\PYZhy{}}\PY{n}{y1}\PY{o}{*}\PY{n}{L1}  \PY{o}{\PYZhy{}} \PY{n}{y2}\PY{o}{*}\PY{n}{L2}   \PY{o}{+}   \PY{n}{y3} \PY{o}{*} \PY{n}{L2}   \PY{o}{+}   \PY{n}{y4}\PY{o}{*}\PY{n}{L1}\PY{p}{)}\PY{p}{]}\PY{p}{]}\PY{p}{)}
         \PY{n}{I} \PY{o}{=} \PY{n}{np}\PY{o}{.}\PY{n}{identity}\PY{p}{(}\PY{n}{dof}\PY{p}{)}
\end{Verbatim}

    \begin{Verbatim}[commandchars=\\\{\}]
{\color{incolor}In [{\color{incolor} }]:} 
\end{Verbatim}

    \begin{Verbatim}[commandchars=\\\{\}]
{\color{incolor}In [{\color{incolor} }]:} 
\end{Verbatim}

    \begin{Verbatim}[commandchars=\\\{\}]
{\color{incolor}In [{\color{incolor} }]:} 
\end{Verbatim}

    O ile znak dodawania + oraz mnożenia * jest taki sam jak w znacznej
większości języków programowania, to szczególną uwage należy zwrócić na
znak podnoszenia do potęgi. W języku Python wykorzystywany jest **
zamiast powszechnie używanego "\^{}".

    Przejdźmy teraz do zdefiniowania pierwszej funkcji w sekcji \#Funkcje
wlasne. Będzie to funkcja opisujaca profil drogi. Na podstawie
parametrów zadanych w czasie wywołania będzie zwracała funkcję opisującą
wymuszenia. Definicja funkcji jest niebywale prosta. Należy jednak
pamiętać o kilku zasadach. Definicję rozpoczynamy od słowa 'def'. Dalej
po spacji podajemy nazwę funkcji, przy pomocy której będziemy ją
wywoływać. Musi spełniać ona ogólne wymogi dotyczące nazw zmiennych oraz
funkcji w języku Python. W nawiasie znajdować się będą parametry, które
zostaną podane w czasie wywołania funkcji. W poniższym przykładzie jest
to zmienna a. Koniecznie należy pamiętać o zakończeniu nagłówka funkcji,
którym jest znak ":". Instrukcje wewnątrz pętli muszą rozpoczynać się od
wcięcia na jeden tabulator (4 spacje). Jest to podstawa składni języka
Python, która odróżnia go od innych języków. Wcięcia w tekście służą do
porządkowania kodu i są wymagane do poprawnego działania programu.
Definicja funkcji, która zwraca jakąś wielkość zakończona jest
poleceniem "return".

    \begin{Verbatim}[commandchars=\\\{\}]
{\color{incolor}In [{\color{incolor}88}]:} \PY{k}{def} \PY{n+nf}{profil\PYZus{}drogi}\PY{p}{(}\PY{n}{a1}\PY{p}{,} \PY{n}{a2}\PY{p}{,} \PY{n}{T1}\PY{p}{,} \PY{n}{T2}\PY{p}{,} \PY{n}{v}\PY{p}{,} \PY{n}{T}\PY{p}{)}\PY{p}{:}
             \PY{c+c1}{\PYZsh{}Zwraca funkcje poliharmoniczna na podstawie wektora czasu}
             \PY{c+c1}{\PYZsh{}Input:    a1,T1 \PYZhy{} amplituda i okres pierwszej skladowej wymuszenia}
             \PY{c+c1}{\PYZsh{}          a2,T2 \PYZhy{} amplituda i okres drugiej skladowej wymuszenia}
             \PY{c+c1}{\PYZsh{}          v     \PYZhy{} predkosc pojazdu}
             \PY{c+c1}{\PYZsh{}          ds, dt\PYZhy{} krok dla czasu i dla drogi}
             \PY{c+c1}{\PYZsh{}          T     \PYZhy{} wektor czasu symulacji}
             \PY{n}{result} \PY{o}{=} \PY{n}{np}\PY{o}{.}\PY{n}{zeros}\PY{p}{(}\PY{p}{(} \PY{n}{np}\PY{o}{.}\PY{n}{size}\PY{p}{(}\PY{n}{T}\PY{p}{)}\PY{p}{)}\PY{p}{)}     \PY{c+c1}{\PYZsh{}tworzenie macierzy wynikowej}
             \PY{c+c1}{\PYZsh{}przypisywanie profilu drogi z uwzględnieniem plaskiego poczatku}
             
             
             \PY{k}{for} \PY{n}{i} \PY{o+ow}{in} \PY{n+nb}{range}\PY{p}{(}\PY{n}{np}\PY{o}{.}\PY{n}{size}\PY{p}{(}\PY{n}{T}\PY{p}{)}\PY{o}{\PYZhy{}}\PY{l+m+mi}{1}\PY{p}{)}\PY{p}{:}
                 \PY{k}{if} \PY{p}{(}\PY{n}{T}\PY{p}{[}\PY{n}{i}\PY{p}{]} \PY{o}{\PYZlt{}} \PY{n}{T1}\PY{p}{)}\PY{p}{:}
                     \PY{n}{result}\PY{p}{[}\PY{n}{i}\PY{p}{]} \PY{o}{=} \PY{l+m+mi}{0}
                 \PY{k}{else}\PY{p}{:}
                     \PY{n}{result}\PY{p}{[}\PY{n}{i}\PY{p}{]} \PY{o}{=} \PY{n}{a1}\PY{o}{*} \PY{n}{sin}\PY{p}{(}\PY{l+m+mi}{2}\PY{o}{*} \PY{n}{pi}\PY{o}{/}\PY{n}{T1}\PY{o}{*}\PY{n}{T}\PY{p}{[}\PY{n}{i}\PY{p}{]}\PY{p}{)} \PY{o}{+} \PY{n}{a2}\PY{o}{*} \PY{n}{sin} \PY{p}{(}\PY{l+m+mi}{2}\PY{o}{*}\PY{n}{pi}\PY{o}{/}\PY{n}{T2}\PY{o}{*}\PY{n}{T}\PY{p}{[}\PY{n}{i}\PY{p}{]}\PY{p}{)}
             \PY{k}{return} \PY{n}{result}
\end{Verbatim}

    \begin{Verbatim}[commandchars=\\\{\}]
{\color{incolor}In [{\color{incolor}92}]:} \PY{n}{A} \PY{o}{=} \PY{n}{profil\PYZus{}drogi}\PY{p}{(}\PY{l+m+mi}{2}\PY{p}{,} \PY{l+m+mf}{0.05}\PY{p}{,} \PY{l+m+mi}{10} \PY{p}{,} \PY{l+m+mi}{1}\PY{p}{,} \PY{n}{v}\PY{p}{,} \PY{n}{T}\PY{p}{)}
         \PY{c+c1}{\PYZsh{}plot(T,A)}
\end{Verbatim}

    \begin{Verbatim}[commandchars=\\\{\}]
{\color{incolor}In [{\color{incolor}93}]:} \PY{n}{fig}\PY{p}{,} \PY{n}{ax} \PY{o}{=} \PY{n}{plot}\PY{o}{.}\PY{n}{subplots}\PY{p}{(}\PY{p}{)}
         \PY{n}{ax}\PY{o}{.}\PY{n}{plot}\PY{p}{(}\PY{n}{T}\PY{p}{,} \PY{n}{A}\PY{p}{)}
         \PY{n}{ax}\PY{o}{.}\PY{n}{axis}\PY{p}{(}\PY{l+s+s1}{\PYZsq{}}\PY{l+s+s1}{equal}\PY{l+s+s1}{\PYZsq{}}\PY{p}{)}
         \PY{n}{ax}\PY{o}{.}\PY{n}{set}\PY{p}{(}\PY{n}{xlabel}\PY{o}{=}\PY{l+s+s1}{\PYZsq{}}\PY{l+s+s1}{time (s)}\PY{l+s+s1}{\PYZsq{}}\PY{p}{,} \PY{n}{ylabel}\PY{o}{=}\PY{l+s+s1}{\PYZsq{}}\PY{l+s+s1}{voltage (mV)}\PY{l+s+s1}{\PYZsq{}}\PY{p}{,}
                \PY{n}{title}\PY{o}{=}\PY{l+s+s1}{\PYZsq{}}\PY{l+s+s1}{Profil drogi}\PY{l+s+s1}{\PYZsq{}}\PY{p}{)}
         \PY{n}{ax}\PY{o}{.}\PY{n}{grid}\PY{p}{(}\PY{p}{)}
         
         \PY{c+c1}{\PYZsh{}fig.savefig(\PYZdq{}test.png\PYZdq{})}
         \PY{n}{plot}\PY{o}{.}\PY{n}{show}\PY{p}{(}\PY{p}{)}
\end{Verbatim}

    \begin{center}
    \adjustimage{max size={0.9\linewidth}{0.9\paperheight}}{output_40_0.png}
    \end{center}
    { \hspace*{\fill} \\}
    
    \begin{Verbatim}[commandchars=\\\{\}]
{\color{incolor}In [{\color{incolor} }]:} 
\end{Verbatim}

    \begin{Verbatim}[commandchars=\\\{\}]
{\color{incolor}In [{\color{incolor} }]:} 
\end{Verbatim}


    % Add a bibliography block to the postdoc
    
    
    
    \end{document}
